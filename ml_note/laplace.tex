%
% Copyright(c) 2015 Taikai Takeda <297.1951@gmail.com> All rights reserved.
%
\section{Laplace's method}
\label{sec:laplace}
Laplace's mothod (smoothing) とは,ある分布$q(\bm{\theta})$をGaussianで近似する手法である.
ここでは,Laplace's mothodによる分布の積分の近似を示す.\cite{konishi_kitagawa200409}

パラメータ$\bm{\theta}$の分布に関する積分を考える($n$は標本数)
\begin{equation}
  \label{eq:1}
  \int \exp(n q(\bm{\theta}))d \bm{\theta}
\end{equation}
を考える.$\bm{\theta}$のmodeを$\bm{\hat{\theta}}$とし,その近傍でテイラー展開する.($n$が十分に大きければ$\bm{\theta}$はmodeに集中すると仮定して近似している).
$\left.\frac{\partial q(\bm{\theta})}{\partial \bm{\theta}}\right|_{\bm{\hat{\theta}}}=0$であるので,
\begin{equation}
  \label{eq:2}
  q(\bm{\theta}) =   q(\bm{\hat{\theta}}) -
  \frac{1}{2}(\bm{\theta} - \bm{\hat{\theta}})^T J(\bm{\hat{\theta}}) (\bm{\theta} - \bm{\hat{\theta}}) + o((\bm{\theta} - \bm{\hat{\theta}})^2)
\end{equation}
\begin{itemize}
\item 
\end{itemize}
ここで,
\begin{equation}
  \label{eq:3}
  J(\bm{\hat{\theta}}) = - \left.\frac{\partial^2 q(\bm{\theta})}{\partial \bm{\theta} \partial \bm{\theta}^T}\right|_{\bm{\hat{\theta}}}
\end{equation}
とした.
\begin{equation}
  \label{eq:5}
  \int \exp\left\{-\frac{1}{2}(\bm{\theta} - \bm{\hat{\theta}})^T J(\bm{\hat{\theta}}) (\bm{\theta} - \bm{\hat{\theta}}) \right\}d\bm{\theta} = (2\pi)^{p/2} n^{-p/2} |J(\bm{\hat{\theta}})|^{-1/2}
\end{equation}
より,結局,
\begin{equation}
  \label{eq:6}
    \int \exp(n q(\bm{\theta}))d \bm{\theta} \approx
    (2\pi)^{p/2} n^{-p/2} |J(\bm{\hat{\theta}})|^{-1/2} \exp(nq(\bm{\hat{\theta}}))
\end{equation}
と近似できる.

