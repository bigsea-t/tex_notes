%
% Copyright(c) 2015 Taikai Takeda <297.1951@gmail.com> All rights reserved.
%

\section{Variational  Bayes}
\label{sec:vb}

\def\cL{{\cal L}}
\def\tp{\tilde{p}}

Variational Bayes(VB,変分ベイズ)は,正確に求めることが難しい
潜在変数の事後分布$p(\bm Z|\bm X)$を近似する枠組みである.
同様の目的の手法としてSampling法があるが,計算量の点でVBが有利である.
\\

EMアルゴリズムと同様に,我々の目的は事後分布$p(\bm Z| \bm X)$およびモデルのエビデンス$p(X)$を求めることである.
なお,EMアルゴリズムではパラメータ集合$\bm \theta$を定義したが,ここでは確率変数として$\bm Z$に含まれていることにする.
\begin{eqnarray}
  \ln p(\bm X) &=& \cL(q) + KL(q || p) \\
  \cL(q) &=& \int_{\bm Z}q(\bm Z)\ln\left\{\frac{p(\bm X,\bm Z )}{q(\bm Z)}\right\} d \bm Z\\
  KL(q||p) &=& - \int_{\bm Z}q(\bm Z) \ln\left\{\frac{p(\bm Z | \bm X)}{q(\bm Z)}\right\} d \bm Z
\end{eqnarray}

EMアルゴリズムの場合と同様に,KL距離を最小化する事により最適化を行いたいが,
ここでは真の事後分布$p(\bm Z | \bm X)$を求めるのが不可能であると仮定する.
したがって,代わりにある制限したクラスの$q(\bm Z)$を考え,この中での最適化を考える事になる.

$\bm Z$に関して,$\bm Z$の要素をいくつかのグループに分解し,それらが独立であることを仮定する.
\begin{equation}
  q(\bm Z) = \prod^M_i\bm q_i(Z_i)
\end{equation}
この仮定のもとで,エビデンスの下界$\cL$を最大化することを考える.したがって,$q(\bm Z)$の各因子それぞれに関して,
順番に最適化を行っていく.記法を簡単にするために$q_i(Z_i)$を$q_i$とし,$\cL$を$q_i$の関数とみなす.
\begin{eqnarray}
  \cL(q_j)
  &=& \int_{\bm Z}\prod^M_i q_i \left\{ \ln p (\bm X, \bm Z) - \sum^M_i \ln q_i \right\} d \bm Z \nonumber \\
  &=& \int q_j \left\{ \int \prod_{i \neq j} q_i \ln p (\bm X, \bm Z) d \bm Z_i \right\} d \bm Z_j
      - \int \prod_i q_i \sum_i \ln q_i d \bm Z + const \nonumber \\
  &=& \int q_j \left\{ \int \prod_{i \neq j} q_i \ln p (\bm X, \bm Z) d \bm Z_i \right\} d \bm Z_j
      - \sum_i \int \prod_i q_i \ln q_i d \bm Z + const \nonumber \\
  &=& \int q_j \left\{ \int \prod_{i \neq j} q_i \ln p (\bm X, \bm Z) d \bm Z_i \right\} d \bm Z_j
      - \int q_j \ln q_j d \bm Z_j + const \nonumber \\
  &=& \int q_j \ln \tp(\bm X, \bm Z_j) d \bm Z_j - \int q_j \ln q_j d \bm Z_j + const
  \label{eq:independent_L}
\end{eqnarray}
ここで,新しい分布
\begin{equation}
\tp(\bm X, \bm Z_j) = E_{i \neq j}[\ln p(\bm X, \bm Z)] + const 
\end{equation}
を定義した.\refeq{eq:independent_L}の最大化を考えるが,これは$q_i$と$\tp(\bm X, \bm Z_j)$の
負のKL距離となっていることが分かる.よって,
\begin{equation}
  \ln q_i^* = E_{i \neq j}[\ln p(\bm X, \bm Z)] + const 
\end{equation}
のときに$\cL$は最大になる.なお,定数項を得るためには分布を正規化すれば良い.


